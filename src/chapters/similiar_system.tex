Технологийн хөгжил болон ковид-19 цар тахлаас үүдэн дэлхийн хэмжээнд онлайн худалдааны сайтуудын борлуулалт, худалдан авагч болон онлайн дэлгүүрүүдийн тоо өндөр өсөлттэй байна.

Үүнийгээ дагаад маш олон тооны интернэтээр бараа бүтээгдэхүүн борлуулдаг (e-commerce) төрлийн сайтууд үүссэн ба эдгээрээс манай бүтээгдэхүүнтэй ижил төстэй 2 системийг авч танилцууллаа.

\section{Shopify}
Shopify бол бизнес эрхлэгчдэд онлайн дэлгүүр үүсгэх, удирдах боломжийг олгодог цахим худалдааны платформ юм. Энэ нь хувь хүмүүс болон бизнесүүдэд техникийн өргөн мэдлэг шаардлагагүйгээр өөрсдийн онлайн дэлгүүрээ байгуулж, ажиллуулахад туслах олон төрлийн хэрэгсэл, үйлчилгээгээр хангадаг. Shopify-ийн гол онцлогууд нь:
\begin{itemize}
   \item \textbf{Дэлгүүрийн дизайн:} Хэрэглэгчид дэлгүүрийнхээ өнгө үзэмжийг өөрийн брэндэд тохируулан өөрчлөх боломжтой.

   \item  \textbf{Бүтээгдэхүүний менежмент:} Борлуулагчид бүтээгдэхүүнээ хялбархан нэмж, тайлбар, үнэ, бараа материалын түвшин зэрэг дэлгэрэнгүй мэдээллийг оруулах боломжтой.

   \item  \textbf{Төлбөрийн боловсруулалт:} Shopify нь төлбөрийн олон шийдлийг дэмждэг тул үйлчлүүлэгчдэд төлбөрийн янз бүрийн аргыг ашиглан худалдан авалт хийхэд хялбар болгодог.

   \item  \textbf{Аюулгүй байдал:} Shopify нь хостинг, төлбөр хийх, хэрэглэгчийн мэдээллийг хамгаалах зэрэг аюулгүй байдлын асуудлыг хариуцдаг.

   \item \textbf{Захиалгын менежмент:} Энэхүү платформ нь захиалгыг удирдах, бараа материалыг хянах, тээвэрлэлт, гүйцэтгэлийг зохицуулахад тусалдаг.

   \item \textbf{Аналитик:} Бизнесүүдэд туслах зорилгоор борлуулалт, хэрэглэгчийн зан төлөв болон бусад чухал хэмжүүрүүдийг хянах аналитик хэрэгслээр хангадаг.
\end{itemize}


\section{Saleor}
Saleor нь бизнес эрхлэгчдэд онлайн дэлгүүр байгуулах, удирдах боломжийг олгодог нээлттэй эхийн цахим худалдааны платформ юм. Уян хатан байдал, өргөтгөх боломжтой, ашиглахад хялбар болгох үүднээс эдгээр технологийг ашигладаг. Saleor нь өөрсдийн цахим худалдааны вэбсайтыг бий болгохыг хүсэж буй хөгжүүлэгчид болон бизнесүүдэд зориулсан.

Ашигласан технологийн хувьд яг ижил ба үүнд React, GraphQL, болон Django технологиудыг ашиглан хөгжүүлжээ. Мөн Merchant вебийнхээ интерфейс дээр Material-ui хэрэглэсэн байна. Хэрэглэгч буюу дэлгүүрийн админ өөрийн хүссэн үедээ шинээр дэлгүүр нээх боломжтой. Ингэснээр зөвхөн танай дэлгүүрт зориулагдсан e-commerce веб, дэлгүүрээ удирдах мерчант вебийг бэлдэн гаргаж өгөх болно. Манай Oneline төсөл дэлгүүр хэсгээ зөвхөн нэг гар утасны апп дээр шийдсэн бол уг систем веб болон PWA ашиглан дэлгүүр бүрд шинэ бүтээгдэхүүн өгдөг байдлаараа ялгаатай байна.