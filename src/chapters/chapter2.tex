\chapter{Системийн шинжилгээ, зохиомж}
Системийн шинжилгээ бүлгийн хүрээнд өмнөх бүлгээс олж авсан мэдээллийн дагуу системийн функциональ ба функциональ бус шаардлагыг боловсруулж түүнээс системийн статик болон динамик загварыг боловсруулан системийн ерөнхий архитектурыг гаргасан болно.

\section{Системийн шаардлага}
\subsection{Функциональ шаардлагуудыг дараах хүснэгтэд тодорхойлов}
\begin{table}[h!]
	\centering
   \begin{tabularx}{\textwidth}{|p{0.1\textwidth}|X|}
		\hline
      ФШ 10 & Систем нь цахим бүтээл болон лицензийн талаарх мэдээллийн найдвартай байдлыг хадгалахын тулд блокчэйнтэй харилцах ёстой.
      \\ \hline ФШ 20 & Системд хэрэглэгчид крипто хэтэвчээ ашиглан өөрийгөө баталгаажуулах боломжтой байх ёстой (жишээ нь, MetaMask).
      \\ \hline ФШ 30 & Системд зөвхөн холбогдсон түрийвчтэй, баталгаажуулсан хэрэглэгчид системийн функцэд хандах эрхтэй байх ёстой.
      \\ \hline ФШ 40 & Систем нь хэрэглэгчдэд янз бүрийн төрлийн цахим бүтээлийг (жишээ нь, видео, аудио, PDF, зураг) аюулгүйгээр байршуулах боломжтой байх ёстой.
      \\ \hline ФШ 50 & Систем нь цахим бүтээлийг байршуулахдаа бүтээлийн мэдээллийг бүртгэх мөн үнийг тогтоох боломжтой байх ёстой.
      \\ \hline ФШ 60 & Систем нь өгөгдлийн нууцлал, аюулгүй байдлыг хангах үүднээс байршуулахаас өмнө файлуудыг шифрлэн IPFS (InterPlanetary File System) дээр найдвартай хадгалах ёстой.
      \\ \hline ФШ 70 & Систем нь шифрлэгдсэн файлуудыг зохих зөвшөөрөлтэй эрх бүхий хэрэглэгчдэд зориулан тайлах ёстой.
      \\ \hline
	\end{tabularx}
\end{table}

\begin{table}[h!]
   \centering
   \begin{tabularx}{\textwidth}{|p{0.1\textwidth}|X|}
      \hline ФШ 80 & Системд хэрэглэгчдэд цахим бүтээлийн үнэд тохирсон шаардлагатай төлбөрийг төлж эзэмшигчдээс тухайн цахим бүтээлд хандах лиценз буюу хандах зөвшөөрөл авах хүсэлт илгээх боломжтой байх ёстой.
      \\ \hline  ФШ 90 & Системд цахим бүтээл эзэмшигчид лицензийн хүсэлтийг зөвшөөрөх эсвэл татгалзах боломжтой байх ёстой. Зөвшөөрөгдсөн үед худалдан авагчид бүтээлийн лицензийг өгөх мөн зохих төлбөрийг эзэмшигчид шилжүүлэх ёстой.
      \\ \hline ФШ 100 &  Системд цахим бүтээл эзэмшигчид мөн хэрэглэгчдийн лицензийн хүсэлтээс татгалзах сонголттой байх ёстой. Ийм тохиолдолд төлбөрийг хэрэглэгчдэд буцааж өгөх ёстой.
      \\ \hline ФШ 110 &  Систем нь цахим бүтээл эзэмших, түүнд лиценз олгох асуудлыг зохицуулахын тулд ухаалаг гэрээг блокчэйн дээр байршуулж, удирдах ёстой.
      \\ \hline ФШ 120 &  Системд хэрэглэгчид хүссэн үедээ системээс цахим бүтээлийн орлогоо өөрийн крипто хэтэвч рүү татах боломжтой байх ёстой.
      \\ \hline ФШ 130 &  Систем нь бүтээл байршуулах, удирдах, өмчлөлийг баталгаажуулахын тулд ухаалаг гэрээтэй харилцах ёстой.
      \\ \hline ФШ 140 &  Хэрэглэгчид cистем дээр байршуулсан цахим бүтээлийн дэлгэрэнгүй мэдээллийг үзэх боломжтой байх ёстой.
      \\ \hline ФШ 150 &  Цахим бүтээлийн лиценз авахад лицензэд өвөрмөц дугаар олгож, блокчэйн дээр хадгалах ёстой.
      \\ \hline ФШ 160 &  Систем нь хэрэглэгчид лиценз авсны дараа лицензийн дугаар, файлын мэдээлэл зэрэг лицензийнхээ дэлгэрэнгүй мэдээллийг агуулсан цахим гэрчилгээ олгох ёстой.
      \\ \hline
	\end{tabularx}
   \caption{Функциональ шаардлагууд}
\end{table}
\clearpage

\subsection{Функциональ бус шаардлагуудыг дараах хүснэгтэд тодорхойлов}
\begin{table}[h!]
   \centering
   \begin{tabularx}{\textwidth}{|p{0.1\textwidth}|X|}
   \hline ФБШ 10 & Блокчэйн технологи нь өгөгдлийн бүрэн бүтэн байдлыг хангаж, лицензийн мэдээллийг зөвшөөрөлгүй өөрчлөхөөс сэргийлнэ.
   \ \\ \hline ФБШ 20 & Систем нь гүйцэтгэлийн бууралтгүйгээр олон тооны хэрэглэгчид болон лицензүүдийг зохицуулах чадвартай байх ёстой.
   \ \\ \hline ФБШ 30 & Ухаалаг гэрээ нь модульчлагдсан байх ёстой бөгөөд шинэчлэгдэхэд хялбар байх ёстой.
   \ \\ \hline ФБШ 40 & Систем нь хүлээн зөвшөөрөгдсөн тодорхой хугацааны дотор баталгаажуулах хүсэлтийг хурдан боловсруулах чадвартай байх ёстой.
   \ \\ \hline ФБШ 50 & Систем нь янз бүрийн техникийн чадвартай хэрэглэгчдэд үүнийг үр дүнтэй ашиглах боломжийг олгодог хэрэглэгчдэд ээлтэй интерфейстэй байх ёстой.
   \ \\ \hline ФБШ 50 & Систем нь янз бүрийн үйлдлийн систем, хөтөч, төхөөрөмжтэй нийцтэй байх ёстой.
   \ \\ \hline ФБШ 60 & Систем нь лиценз олгох, дижитал гүйлгээ, блокчэйн технологитой холбоотой аливаа зохицуулалтын шаардлагад нийцэж байх ёстой.
   \ \\ \hline ФБШ 70 & Энэ систем нь гамшгийн үед өгөгдөл алдагдахгүй байхын тулд найдвартай нөөцлөх, сэргээх механизмтай байх ёстой.
   \\ \hline
\end{tabularx}
\caption{Функциональ бус шаардлагууд}
\end{table}
\clearpage

\section{Системийн ажлын явцын диаграмм}
Цахим бүтээлийн ажлын явцыг дараах байдлаар тодорхойлов. Системийн оролцогч талуудыг цахим бүтээлийн лиценз буюу хандах зөвшөөрөл авах гэж буй хэрэглэгч. Нөгөө талаас цахим бүтээл оруулах, түүнд лиценз буюу хандах зөвшөөрөл олгож буй бүтээл эзэмшигч гэж тодорхойлсон.
\begin{figure}[h!]
	\centering
	\includegraphics[scale=0.36]{src/images/usecase.png}
	\caption{Use-case диаграмм}
\end{figure}

\newpage
\pagebreak
\section{Дарааллын диаграмм}
\subsection{Хэрэглэгч цахим бүтээл оруулах дарааллын диаграмм}
Энэхүү дарааллын диаграмм нь цахим бүтээлийг төвлөрсөн бус системд байршуулах, Lit протокол ашиглан шифрлэх, IPFS сүлжээнд хадгалах, блокчэйн дээр бүртгэх үйл явцыг дүрсэлсэн болно.

\begin{figure}[h!]
	\centering
	\includegraphics[scale=0.28]{src/images/sequence.png}
	\caption{Хэрэглэгч цахим бүтээл оруулах дарааллын диаграмм}
\end{figure}

\pagebreak
\subsection{Хэрэглэгч цахим бүтээлд хандах эрх хүсэх дарааллын диаграмм}
Энэхүү дарааллын диаграмм нь эзэмшигчээс цахим бүтээлийн лиценз авах хүсэлт гаргасан хэрэглэгч, дараа нь хүсэлтийг зөвшөөрөх эсвэл татгалзах үйл явцыг дүрсэлсэн болно.
\begin{figure}[h!]
	\centering
	\includegraphics[scale=0.6,]{src/images/sequence-2.png}
	\caption{Хэрэглэгч цахим бүтээлийг хүсэх дарааллын диаграмм}
\end{figure}

\newpage
\section{Архитектур}
Энэхүү төслийн фронтэнд хэсэг нь NextJS-н ашигласан тул сервер талын рендер хийж байгаа ба хэрэглэгчийн оруулсан цахим бүтээл болон лицензийн мэдээллийг этереум блокчэйн сүлжээнд байршсан ухаалаг гэрээнд бичих болон унших үйлдлийг хийх юм. Мөн хэрэглэгчийн оруулсан цахим бүтээлийн файлыг IPFS сүлжээнд шифрлэн байршуулж, шифрлэлтийн түлхүүрийг Lit сүлжээнд хадгална. Шифрлэлтийн түлхүүрийг хандалтын хяналтын нөхцөлд тодорхойлсны дагуу ухаалаг гэрээгээр зөвшөөрөлтэй эсэхийг шалган авч файлын шифрлэлтийг тайлна. \\

\begin{figure}[h!]
	\centering
	\includegraphics[scale=0.26]{src/images/architecture.png}
	\caption{Системийн ерөнхий архитектур}
\end{figure}
