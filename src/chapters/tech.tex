\section{Git болон GitHub}
Linus Trovalds буюу Linux Kernel-г хөгжүүлсэн хүн Kernel-ийнхээ эх кодыг удирдах зорилгоор уг технологийг анх санаачилж, хэрэгжүүлсэн байдаг. Гол зорилго нь Version Control System буюу хувилбар удирдах системийг бүтээх ба ингэснээр хөгжүүлэлтийн явцад бүх өөрчлөлтөө хадгалах, багаар ажиллах боломжийг бүрдүүлж өгсөн юм.

Үндсэн ажиллагааг нь тайлбарлавал Git ашиглаж буй хөгжүүлэгч/хэрэглэгч бүр өөрийн төхөөрөмж дээр үндсэн repository-тай яг ижил repository-г үүсгэнэ. Ингэснээр сүлжээнд холбогдсон эсэхээс үл хамааран дурын өөрчлөлт, хөгжүүлэлтээ хийх боломжтой ба уг ажил дууссан үед бичсэн өөрчлөлтүүдээ commit хийж, remote repository руу push үйлдлийг хийнэ.

GitHub бол Git-г ашигладаг веб платформ юм. Бусадтай хамтарч ажиллах, төслөө хадгалах, прожект менежмент, continuous integration, болоод бусдын төслийг харах, судлах, хуулбарлах гэх мэт маш олон үйлчилгээг үзүүлдэг. Гол онцлогуудаас дурдвал,

\begin{itemize}
	\item Гит репозиторуудыг хадгалж, тэдгээртэй харьцахад хялбар UI.
	\item Багаар ажиллаж байхдаа нэг нэгнийхээ кодыг хянах байдлаар хийсэн өөрчлөлтүүдийг бусад гишүүдээр эсвэл хариуцсан мэргэжилтнээр нь хянуулж бусад мөчиртэй нэгтгэх хүсэлт илгээж болдог.
	\item Forks: Төсөлд ямар ч нөлөө үзүүлэхгүйгээр хуулбарлаж хөгжүүлэлт хийх боломж.
	\item Issues: Алдаа, bug болон сайжруулалтыг хянах нэг арга.
	\item Actions: Хөгжүүлэх, тестлэх, байршуулах ажлын урсгалыг автоматжуулдаг.
\end{itemize}

\section{Javascript}
\quad JavaScript нь прототайпад суурилсан скриптийн хэл бөгөөд динамик, өгөгдлийн сул төрөлжүүлэлттэй, нэгдүгээр зэргийн функцүүдтэй хэл юм. Энэ нь объект хандлагат, исператив, функциональ програмчлалын хэв маягуудыг дэмждэг олон парадигмт хэл юм.

Ажиллагааны хувьд товчхондоо JavaScript engine нь JS code-г хөрвүүлээд ажиллуулдаг. Анх веб хөтөч дээр ажиллахад зориулж гарч ирсэн тул олон янзын веб хөтчүүдэд ижилхэн ажиллах стандарт хэрэгтэй болсон нь ECMAScript болж л дээ. Бидний түгээмэл хэрэглэдэг стандарт нь ES6 (ECMAScript 6) буюу 2015 оны хувилбар нь.

JavaScript-г ашиглаад дан ганц веб биш мобайл апп, тоглоом эсвэл BackEnd сервер ч бас хийж болдог. React.js, Vue.js ашиглан FrontEnd, Node.js суурьтай Loopback, Nest.js ашиглан BackEnd эсвэл Ionic, React Native ашиглаад гар утасны апп хийж болно.

\section{React}
Фейсбүүк компани дотооддоо ашиглаж байсан технологио 2013 онд танилцуулсан нь Javascript хэлийг ашиглаж хийсэн Front-end library болох React технологи юм. Declarative UI хөгжүүлэлтийн аргыг хамгийн анх дэлгэрүүлж, өргөн хэрэглээнд нэвтрүүлж чадсан тул Declarative UI-н гол төлөөлөгч гэж явдаг. Уг технологийг ашиглахын тулд үндсэн хэдэн ойлголтууд авах хэрэгтэй. Үүнд component ба түүний lifecycle, javascript-н өргөжүүлсэн хувилбар болох jsx, мөн хамгийн чухал зүйл болох Virtual DOM нар багтана.

Declarative UI гэдэг нь хэрэглэгчийн интерфейсийн кодыг бичихдээ юу зурагдах буюу render хийх үеийн интерфейсийг бүгдийг урьдчилан тодорхойлдог. Imperative програмчлалаас ялгаатай нь хязгаартай нөхцөлд яг юу хийхийг хатуугаар зааж өгөхгүйгээр тухайн state-с хамааруулж хэрэглэгчийн хүссэн зүйлийг гаргаж өгөх боломжтой.

React нь component-based буюу DOM дээр хэвлэж байгаа бүх зүйлс component байна гэсэн дүрмийг баримталдаг. Component үүсгэж бичихийн давуу тал нь нэг бичсэн кодоо олон дахин бичигдэхээс зайлсхийж, дахин ашиглах боломжийг олгодог. Тус бур өөрсдийн гэсэн дотоод төлөвтэй мөн гаднаас утга хүлээн авах чадвартай. Үүнийг бид Props гэж нэрлэдэг. Мөн component нь stateless, stateful гэж хоёр хуваагддаг ба stateful component нь өөрийн гэсэн төлөвтэй, түүнийгээ удирддаг, class болон hook ашигласан функцүүд байна. React-н давуу тал нь state эсвэл props-н өөрчлөлтийг үргэлж хянаж байдаг тул өөрчлөлт орж ирэхэд бүтэн хуудсыг зурах бус зөвхөн тухайн өөрчлөгдсөн component-г л дахин зурдаг. Ингэснээр энгийн вебүүдээс илүү хурдтай ажилладаг.

JSX нь Javascript Extended гэсэн үгний товчлол бөгөөд энгийнээр javascript дотор HTML-н тагуудыг бичиж өгөх мөн кодыг илүү богино болгож хүссэн үр дүндээ хүрэх боломжийг олгодог. Үүний цаана Babel гэсэн transcompiler-г ашиглаж дундын хөрвүүлэлтийг хийдэг ба хэдийгээр HTML таг бичиж байгаа харагддаг ч код дунд цэвэр HTML-г огтоос бичиж өгдөггүй гэсэн үг юм.

\begin{lstlisting}[caption=JSX ашиглаж ”container” класстай html элемент буцаах компонент, frame=single]
   export default function Container({ children }) {
      return (
         <div className ="container">
            { children }
         </div >
      )
   }
\end{lstlisting}

Жинхэнэ DOM дээр богино хугацаанд олон өөрчлөлт хийхэд удах асуудал гарсан тул React маань Virtual DOM гэсэн abstraction давхарга үүсгэж өөрчлөлтүүдээ Virtual DOM дээрээ хадгалаад нэгдсэн нэг өөрчлөлтийг жинхэнэ DOM руугаа дамжуулдаг.

\section{React Hook Form}
React Hook Form нь React аппликейшнд дахь форм удирдах сан юм. Энэ нь функциональ бүрэлдэхүүн хэсгүүдийн төлөв байдал болон гаж нөлөөг удирдах илүү хялбар арга замаар React 16.8-д нэвтрүүлсэн React hook дээр бүтээгдсэн. React Hook Form нь тусгайлан useForm болон useField hook ашиглан форм хөгжүүлэлтийг илүү үр ашигтай, уян хатан болгодог.

\section{Bootstrap}
2010 онд Twitter-т ажилладаг 2 залуу нийлж, тухайн үед жинхэнэ толгойны өвчин болоод байсан вебийн интерфейс хөгжүүлэх үйл явцыг хөнгөвчлөх зорилготой бяцхан төсөл эхлүүлсэн нь өдгөө дэлхий дээрх хамгийн олон хэрэглэгчидтэй, хамгийн өргөн ашиглагддаг CSS фрэймворк болох Bootstrap байсан юм.

Өнөөдөр тэрхүү бяцхан төсөл цар хүрээгээрээ асар том болж хөгжжээ. Анх Twitter Blueprint нэртэй байсан Bootstrap нь дэлхий дээрх хамгийн анхны CSS фрэймворкуудын нэг билээ. Бусдаас ялгарах гол давуу тал нь илүү найдвартай ажиллагаа, тасралтгүй хөгжүүлэлт байв. Одоо ч гэсэн хамгийн их хөгжүүлэлт хийгддэг, support сайтай фрэймворк хэвээрээ байна.

BuiltWith сайтын судалгаанаас харахад интернэтэд одоо ашиглагдаж буй нийт веб сайтуудын 20.6 хувийг, Javascript library-бүхий вебүүдийн 26.9 хувийг Bootstrap ашиглан хийсэн байдаг бөгөөд нийт фрэймворкийн зах зээлийн 72 хувийг дангаараа эзэлсээр. Мянган мянган алдартай компаниуд энэхүү фрэймворкийг ашигладаг бөгөөд тэдний тоонд Linkedin, Twitter, Spotify, Snapchat, Udemy, Baidu, Zoom, Netflix компаниуд ч багтаж байна.

\section{GitBook}
GitBook нь код хэлбэрээр бичиг баримт боловсруулах, мэдлэг солилцох ажлыг хялбаршуулдаг мэдлэгийн удирдлагын хэрэгсэл юм. Git болон Markdown ашиглан баримт бичиг бүтээх боломжийг олгодог бөгөөд мөн GitHub болон GitLab-тай интеграци хийх боломжийг олгодог. Энэхүү платформыг Adobe, Netflix, Apple, Snyk, Google зэрэг компаниуд зэрэг 2 сая гаруй хэрэглэгчид, олон мянган баг ашигладаг.

% \section{Webflow}
% Webflow нь код бичих ур чадвар шаарддаггүй ашиглахад хялбар веб сайт бүтээх платформ юм. Webflow-ийн тусламжтайгаар та өөрөө ямар ч код бичих шаардлагагүйгээр блог, портфолио, цахим худалдааны дэлгүүр болон бусад төрлийн вебсайт бүтээх боломжтой.