\chapter{Системийн хэрэгжүүлэлт}
Системийн хэрэгжүүлэлт бүлгийн хүрээнд системийн хөгжүүлэлтийг дээрх бүлэг сэдвүүд дээр тодорхойлсон шинжилгээ ба зохиомжийн дагуу хийлээ.

\section{Хөгжүүлэлт}
\subsection{Хөгжүүлэлтийн орчныг бэлдэх}
Энэхүү судалгааны ажлын практик хэсэгт би NextJS, Hardhat, Pinata IPFS, Lit Protocol, Wagmi, Tailwind CSS зэргийг ашиглан хөгжүүлэлт хийх билээ. NextJS нь монолитик төсөл хийхэд тохиромжтой ба төслийн ухаалаг гэрээн хөгжүүлэлт, клайнт талуудыг нэг repository-д хадгалж байгаа. Version Control System-ээр Github-г сонгосон юм. Кодын фолдер бүтэц нь дараах байдлаар байна.

\begin{figure}[htbp]
   \centering
   \begin{minipage}[ht]{0.4\textwidth}
       \centering
       \includegraphics[scale=0.25]{src/images/folder-structure.png}
       \caption{Хөгжүүлэлтийн орчин}
   \end{minipage}
   \begin{minipage}[h!t]{0.5\textwidth}
      \raggedright
      \begin{itemize}
         \item \textbf{components} - React компонентууд
         \item \textbf{lib} - Хэрэглэгчийн талын шаардлагатай код туслах функцүүд
         \item \textbf{app} - NextJS дээрх хуудаснууд
         \item \textbf{public} - Статик зураг, файлууд
         \item \textbf{scripts} - Ухаалаг гэрээний хөгжүүлэлттэй холбоотой файлууд
         \item \textbf{contracts} - Ухаалаг гэрээний файлууд
      \end{itemize}
  \end{minipage}
\end{figure}

\newpage
\subsection{Ухаалаг гэрээн хөгжүүлэлт}
Миний төсөл нэг ухаалаг гэрээнээс бүтнэ. Ухаалаг гэрээг Solidity хэл дээр бичсэн бөгөөд Ethereum  блокчэйн дээр байршуулсан. Уг ухаалаг гэрээ нь цахим файлууд болон тэдгээртэй холбоотой лицензүүдийг төлөөлдөг Файл, Лиценз ба Хүсэлт гэсэн гурван бүтцийг тодорхойлсон. Файл бүтэц  нь id, эзэмшигчийн хаяг, файлын нэр, дэлгэрэнгүй, ангилал, шифрлэсэн файлын хэш, файлын хэмжээ, бүтээлийн үнэ, үүсгэсэн хугацаа зэрэг атрибутуудыг агуулна. Лиценз бүтэц нь лицензийн дугаар, бүтээл эзэмшигчийн хаяг, лиценз эзэмшигчийн хаяг, файлын нэр, дэлгэрэнгүй, ангилал, шифрлэсэн файлын хэш, файлын хэмжээ, лиценз үүсгэсэн хугацаа  зэрэг атрибутуудыг агуулна.
Мөн дараах функцүүдтэй:

\begin{table}[h!]
	\centering
   \begin{tabularx}{\textwidth}{|p{0.35\textwidth}|X|}
		\hline
		 \textbf{createFile}& Цахим бүтээлийн мэдээллийг бичих
	\\ \hline \textbf{getAllPublicFiles} & Оруулсан бүх цахим бүтээлийн мэдээллийг авах
	\\ \hline \textbf{getAllUserFiles} &  Хэрэглэгчийн оруулсан цахим бүтээлийн мэдээллийг авах
	\\ \hline \textbf{getAllUserLicenses} & Хэрэглэгчийн эзэмшиж буй цахим бүтээлийн лицензүүдийг авах
	\\ \hline \textbf{getPublicFileById} & Цахим бүтээлийн мэдээллийг дугаараар нь авах
	\\ \hline \textbf{requestLicense} & Цахим бүтээлийг авах бүтээлийн эзэмшигч рүү лицензийн хүсэлт илгээх
	\\ \hline \textbf{approveLicenseRequest} & Өөрийн цахим бүтээлд ирсэн хүсэлтийг зөвшөөрөх
	\\ \hline \textbf{rejectLicenseRequest} & Өөрийн цахим бүтээлд ирсэн хүсэлтээс татгалзах
	\\ \hline \textbf{getFileOwnerLicenseRequests} & Бүтээлийн эзэмшигчид ирсэн хүсэлтүүдийг авах
	\\ \hline \textbf{isFileOwnedOrLicensed} & Цахим бүтээлийн эзэмшигч эсэх эсвэл түүний лицензийг авсан эсэхийг шалгах
	\\ \hline \textbf{generateUniqueLicense} & Лицензэд өвөрмөц дугаар бий болгох                                                             \\ \hline
	\end{tabularx}
   \caption{Ухаалга гэрээний функцүүд}
\end{table}

\newpage
\subsection{Ухаалаг гэрээг этереум блокчэйн сүлжээнд байршуулах}
\lstinputlisting[language=TypeScript,caption=Ухаалаг гэрээг блокчэйнд байршуулах,basicstyle=\linespread{0.6}\ttfamily,frame=single]{src/code/deploy.js}

\subsection{Lit protocol-н файл шифрлэлт болон хандалтын хяналтын нөхцөл}
Lit Protocol-оор дамжуулан блокчэйнд байршуулсан ухаалаг гэрээнд бичсэн isFileOwnedOrLicensed функц хэрэглэгчийн крипто хэтэвчийн хаяг, файлын дугаараар файлыг зөвхөн бүтээл эзэмшигч эсвэл тухайн бүтээлд лиценз буюу хандах зөвшөөрөлтэй хэрэглэгч эсэхийг баталгаажуулан файлын шифрлэлтийг тайлах түлхүүрийг авах боломжтойгоор хандалтын хяналтын нөхцөлийг тодорхойлон клиент талд файлыг шифрлэнэ. Файлыг шифрлэсний дараа Lit сүлжээ нь хандалтын хяналтын нөхцөлийн тодорхойлсноор хэн шифрлэлтийг тайлах түлхүүр авахыг зөвшөөрнө. Файлын шифрлэлтийн түлхүүр үүсгэх болон авахын тулд хэрэглэгчийн крипто хэтэвчээр баталгаажуулна.

\lstinputlisting[language=TypeScript,caption=Lit protocol-н файл шифрлэлт болон хандалтын хяналтын нөхцөл,basicstyle=\linespread{0.6}\ttfamily,frame=single]{src/code/lit.ts}


\subsection{Pinata API-ээр дамжуулан файлыг IPFS-д хадгалах}
Pinata нь IPFS дээрх үйлдлүүдийг хийх боломжийг олгодог бөгөөд Pinata API-г ашиглан хэрэглэгч контент хадгалах боломжтой болно. Pinata API-г ашиглахын тулд хэрэглэгч Pinata дээр бүртгүүлж, API түлхүүр үүсгэх шаардлагатай.

Pinata API ашиглан файлыг IPFS-д байршуулах функцийг тодорхойлсон. Энэ нь POST хүсэлтийг файлын өгөгдөл болон шаардлагатай аюулгүй байдлын header мэдээллийн хамт илгээж, серверээс хариу хүлээн авсны дараа өгөгдлийг JSON форматаар буцаана.
\lstinputlisting[language=TypeScript,caption=Файлыг IPFS-д хадгалах,basicstyle=\linespread{0.6}\ttfamily,frame=single]{src/code/uploadIpfs.ts}

\subsection{Ухаалаг гэрээнээс өгөгдлийг унших}
Хэрэглэгчийн ethereum хэтэвчний хаяг, ухаалаг гэрээний хаяг болон ухаалаг гэрээний ABI (ухаалаг гэрээ ба ухаалаг гэрээнүүдтэй харилцах программуудын хоорондын харилцааг тодорхойлдог интерфейс) авч, Wagmi сангийн ухаалаг гэрээнээс өгөгдөл авахад зориулагдсан useReadContract hook-г ашиглана. Ethereum блокчэйн дээр байрласан ухаалаг гэрээнээс өгөгдөл авах бөгөөд гэрээ нь 'getAllUserFiles' гэсэн нэртэй функцтэй байх ёстой. Энэ функц нь тухайн хэрэглэгчийн эзэмшдэг цахим бүтээлүүдийг буцаана.
\lstinputlisting[language=TypeScript,caption=Ухаалаг гэрээний өгөгдлийг унших,basicstyle=\linespread{0.6}\ttfamily,frame=single]{src/code/getUserFiles.ts}

\subsection{Ухаалаг гэрээнд цахим бүтээлийн өгөгдлийг бичих}
Wagmi-н тодорхойлсон React Hook-г ашиглан цахим бүтээл эзэмшигчийн оруулсан бүтээлийн өгөгдлийг ухаалаг гэрээнд бичнэ.
Товчхондоо, файл байршуулж, нууцлал, Ethereum ухаалаг гэрээтэй харилцах үйл ажиллагааг React компонент дотор хийнэ. Энэ нь хэрэглэгч Ethereum түрийвчтэй холбогдсон эсэхийг баталгаажуулж, Lit Protocol ашиглан файлыг шифрлэж, шифрлэгдсэн файлыг IPFS-д байршуулж, ухаалаг гэрээнд өгөгдлийг бичнэ. Уг үйл явцын туршид төлөвүүдийн мэдээллийг хэрэглэгчдэд шинэчлэн мэдэгдэнэ.
\lstinputlisting[language=TypeScript,caption=Ухаалаг гэрээний өгөгдлийг унших,basicstyle=\linespread{0.6}\ttfamily,frame=single]{src/code/writeFile.ts}


\newpage
\section{Үр дүн}
Хөгжүүлэлтийг дээрх бүлэг сэдвүүд дээр тодорхойлсон шинжилгээ, зохиомжийн дагуу хийсэн болно. Системийн интерфейс дараах байдлаар харагдана.
\begin{figure}[h!]
	\centering
	\includegraphics[scale=0.15]{src/images/dashboard.png}
	\caption{Хэрэглэгчийн үндсэн хуудас}
\end{figure}

\begin{figure}[h!]
	\centering
	\includegraphics[scale=0.15]{src/images/upload.png}
	\caption{Цахим бүтээл оруулах хуудас}
\end{figure}

\begin{figure}[h!]
	\centering
	\includegraphics[scale=0.16]{src/images/drive.png}
	\caption{Миний сан хуудас}
\end{figure}

\begin{figure}[h!]
	\centering
	\includegraphics[scale=0.16]{src/images/pdf-viewer.png}
	\caption{PDF файл харах}
\end{figure}

\begin{figure}[h!]
	\centering
	\includegraphics[scale=0.16]{src/images/drive.png}
	\caption{Миний сан хуудас}
\end{figure}

\begin{figure}[h!]
	\centering
	\includegraphics[scale=0.16]{src/images/marketplace.png}
	\caption{Marketplace хуудас}
\end{figure}

\begin{figure}[h!]
	\centering
	\includegraphics[scale=0.16]{src/images/marketplace-file.png}
	\caption{Marketplace дэх бүтээлийн хуудас}
\end{figure}

\begin{figure}[h!]
	\centering
	\includegraphics[scale=0.16]{src/images/requests.png}
	\caption{Лиценз хүсэлтийн  хуудас}
\end{figure}