\section{Javascript}
\quad JavaScript нь прототайпад суурилсан скриптийн хэл бөгөөд динамик, өгөгдлийн сул төрөлжүүлэлттэй, нэгдүгээр зэргийн функцүүдтэй хэл юм. Энэ нь объект хандлагат, исператив, функциональ програмчлалын хэв маягуудыг дэмждэг олон парадигмт хэл юм.

Ажиллагааны хувьд товчхондоо JavaScript engine нь JS code-г хөрвүүлээд ажиллуулдаг. Анх веб хөтөч дээр ажиллахад зориулж гарч ирсэн тул олон янзын веб хөтчүүдэд ижилхэн ажиллах стандарт хэрэгтэй болсон нь ECMAScript болж л дээ. Бидний түгээмэл хэрэглэдэг стандарт нь ES6 (ECMAScript 6) буюу 2015 оны хувилбар нь.

JavaScript-г ашиглаад дан ганц веб биш мобайл апп, тоглоом эсвэл BackEnd сервер ч бас хийж болдог. React.js, Vue.js ашиглан FrontEnd, Node.js суурьтай Loopback, Nest.js ашиглан BackEnd эсвэл Ionic, React Native ашиглаад гар утасны апп хийж болно.

\section{React}
Фейсбүүк компани дотооддоо ашиглаж байсан технологио 2013 онд танилцуулсан нь Javascript хэлийг ашиглаж хийсэн Front-end library болох React технологи юм. Declarative UI хөгжүүлэлтийн аргыг хамгийн анх дэлгэрүүлж, өргөн хэрэглээнд нэвтрүүлж чадсан тул Declarative UI-н гол төлөөлөгч гэж явдаг. Уг технологийг ашиглахын тулд үндсэн хэдэн ойлголтууд авах хэрэгтэй. Үүнд component ба түүний lifecycle, javascript-н өргөжүүлсэн хувилбар болох jsx, мөн хамгийн чухал зүйл болох Virtual DOM нар багтана.

Declarative UI гэдэг нь хэрэглэгчийн интерфейсийн кодыг бичихдээ юу зурагдах буюу render хийх үеийн интерфейсийг бүгдийг урьдчилан тодорхойлдог. Imperative програмчлалаас ялгаатай нь хязгаартай нөхцөлд яг юу хийхийг хатуугаар зааж өгөхгүйгээр тухайн state-с хамааруулж хэрэглэгчийн хүссэн зүйлийг гаргаж өгөх боломжтой.

React нь component-based буюу DOM дээр хэвлэж байгаа бүх зүйлс component байна гэсэн дүрмийг баримталдаг. Component үүсгэж бичихийн давуу тал нь нэг бичсэн кодоо олон дахин бичигдэхээс зайлсхийж, дахин ашиглах боломжийг олгодог. Тус бур өөрсдийн гэсэн дотоод төлөвтэй мөн гаднаас утга хүлээн авах чадвартай. Үүнийг бид Props гэж нэрлэдэг. Мөн component нь stateless, stateful гэж хоёр хуваагддаг ба stateful component нь өөрийн гэсэн төлөвтэй, түүнийгээ удирддаг, class болон hook ашигласан функцүүд байна. React-н давуу тал нь state эсвэл props-н өөрчлөлтийг үргэлж хянаж байдаг тул өөрчлөлт орж ирэхэд бүтэн хуудсыг зурах бус зөвхөн тухайн өөрчлөгдсөн component-г л дахин зурдаг. Ингэснээр энгийн вебүүдээс илүү хурдтай ажилладаг.

JSX нь Javascript Extended гэсэн үгний товчлол бөгөөд энгийнээр javascript дотор HTML-н тагуудыг бичиж өгөх мөн кодыг илүү богино болгож хүссэн үр дүндээ хүрэх боломжийг олгодог. Үүний цаана Babel гэсэн transcompiler-г ашиглаж дундын хөрвүүлэлтийг хийдэг ба хэдийгээр HTML таг бичиж байгаа харагддаг ч код дунд цэвэр HTML-г огтоос бичиж өгдөггүй гэсэн үг юм.

\begin{lstlisting}[caption=JSX ашиглаж ”container” класстай html элемент буцаах компонент, frame=single]
   export default function Container({ children }) {
      return (
         <div className ="container">
            { children }
         </div >
      )
   }
\end{lstlisting}

Жинхэнэ DOM дээр богино хугацаанд олон өөрчлөлт хийхэд удах асуудал гарсан тул React маань Virtual DOM гэсэн abstraction давхарга үүсгэж өөрчлөлтүүдээ Virtual DOM дээрээ хадгалаад нэгдсэн нэг өөрчлөлтийг жинхэнэ DOM руугаа дамжуулдаг.