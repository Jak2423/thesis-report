\conclusion{Дүгнэлт}
\indent Бакалаврын судалгааны ажлын хүрээнд блокчэйн технологи болон дижитал эрхийн менежментийн талаар судалсан. Энэхүү судалж суралцсан мэдлэгээ ашиглан практикт цахим бүтээлийн лицензийн системийг бүтээхээр зорьсон. Уг системийн гол зорилго нь хэрэглэгчид блокчэйн технологиор дамжуулан цахим бүтээлийг хамгаалах, хуваалцах, түүнд хандах зөвшөөрөл олгох төвлөрсөн бус, ил тод систем юм.
\\ \indent Энэхүү системийг эхнээс нь алхам алхмаар шаардлагаас нь хэрэгжүүлэлт хүртэл хийж гүйцэтгэсэн ба системийн функциональ ба функциональ бус шаардлагыг боловсруулж түүнээс системийн статик болон динамик загварыг гаргасан билээ. Үр дүнд нь блокчэйн, ухаалаг гэрээ болон IPFS технологиудыг ашиглан найдвартай, ил тод, төвлөрсөн бус системийг бүтээлээ. Түүнчлэн Lit протоколыг ашиглан файлын нууцлалыг сайжруулсан нь платформын аюулгүй байдал, нууцлалыг улам бэхжүүлсэн.
\\ \indent Гэсэн хэдий ч, судалгааны явцад тулгарсан сорилтууд байсан. Нэг том бэрхшээл нь ухаалаг гэрээ боловсруулах болон системд Lit протоколыг нэгтгэхэд шаардлагатай байсан бөгөөд үүний тулд блокчэйн хөгжүүлэлт, криптографын талаар гүнзгий ойлголт шаардлагатай байв. Түүнчлэн, системийн хэмжээнд аюулгүй байдлыг хангахын зэрэгцээ үр ашиг, хурдыг хадгалахад тогтмол сорилт тулгарч байв.
\\ \indent  Цаашид ухаалаг гэрээний функцүүдийг оновчтой болгох, гүйлгээний шимтгэлийг буруулах талаар нарийвчилсан судалгаа хийх нь зүйтэй гэж үзэж байна. Түүнчлэн блокчэйний өөр өөр сүлжээ болон шифрлэлтийн аргуудыг судлах нь системийн цар хүрээ, үр ашгийг дээшлүүлэхэд тустай байж болох юм.