\begin{abstract}
\setcounter{secnumdepth}{0}
Өнөөгийн цахим орчинд зонхилон тохиолдож буй оюуны өмч болон цахим бүтээгдэхүүний хулгай, өмчлөх эрхийн ил тод байдал, зөвшөөрөлгүй түгээлт зэрэг сорилтуудтай тулгарч байна. Блокчэйн технологи нь цахим орчинд аюулгүй байдал, итгэлцлийг хангах, болон төвлөрсөн бус менежментийн боломжийг олгодог онцлог шинж чанартай. Энэ нь зөвхөн криптовалютын салбарт бус, цахим лицензийн баталгаажуулалт, оюуны өмч хамгаалал зэрэг олон салбарт ач холбогдолтой юм. Цахим бүтээгдэхүүний лицензийг баталгаажуулах системд блокчэйн технологийг ашигласнаар хуурамч лиценз, хууль бус хуулбарлалтыг бууруулах, зөвшөөрөлгүй түгээлтийг таслан зогсоох боломжтой.

% Блокчэйн технологийн төвлөрсөн бус, ил тод, хувиршгүй шинж чанарыг ашигласнаар цахим бүтээгдэхүүн эзэмших, лиценз олгоход итгэлцэл, ил тод байдлыг бий болгож, улмаар оюуны өмчийн зөвшөөрөлгүй хулгайн гэмт хэргийг бууруулах үндэслэлээр уг сэдвийг сонгосон.
Блокчэйн технологийн талаар судалгаа анх 2008 онд Сатоши Накамото-гийн “Биткойн: Peer-to-Peer Электрон Мөнгөний Тогтолцоо” нийтлэлээр эхэлсэн. Үүнээс хойш блокчэйн технологийн олон салбар дахь хэрэглээний талаар өргөн хүрээтэй судалгаа хийгдсэн. Гэсэн хэдий ч, цахим лицензийн баталгаажуулалтын тал дээрх судалгаа харьцангуй шинэ бөгөөд энэхүү судалгаагаар энэ чиглэлд тулгарч буй сорилтууд, боломжуудыг илрүүлэн судлах болно.

Энэхүү судалгааны ажлаар блокчэйн технологиор дамжуулан цахим бүтээлийг хамгаалах, хуваалцах, түүнд хандах зөвшөөрөл олгох цахим бүтээлийн лицензийн төвлөрсөн бус систем хөгжүүлэх зорилготой билээ. Уг зорилгын хүрээнд дараах зорилтуудыг тавьсан болно:

\begin{enumerate}
   \item Блокчэйн технологийн талаарх ойлголт, хэрэглээний боломжууд болон дижитал эрхийн менежментийн талаар судлах
   \item Системийг хэрэгжүүлэх шаардлагатай технологи, арга хэрэгслүүдийг судлах
   \item Системийн зохиомж, архитектурыг боловсруулах
   \item Блокчэйн дээр суурилсан цахим бүтээлийн лицензийн систем хөгжүүлэх
\end{enumerate}
\setcounter{secnumdepth}{2}

% Эдгээр зорилтын хүрээнд дараах технологи, арга хэрэгслийг ашиглана:
% \begin{itemize}
%    \item Блокчэйн технологи: Цахим бүтээлийн лицензийн өгөгдлийг бүртгэж, баталгаажуулах.
%    \item Ухаалаг гэрээ (Smart contracts): Лицензийн нөхцөл, хандалтын эрхийг тодорхойлох, хэрэгжүүлэх.
%    \item IPFS (InterPlanetary File System): Цахим бүтээлийг тарааж, хадгалах төвлөрсөн бус систем.
%    \item Lit protocol: Файл шифрлэлт болон хандалтын хяналтын нөхцөлийг хэрэгжүүлэх.
% \end{itemize}

\end{abstract}