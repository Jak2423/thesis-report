%----------------------------------------------------------------------------------------
%   Доорх хэсгийг өөрчлөх шаардлагагүй
%----------------------------------------------------------------------------------------
%!TEX TS-program = xelatex
%!TEX encoding = UTF-8 Unicode
\documentclass[12pt,A4]{report}

\usepackage{fontspec,xltxtra,xunicode}
\setmainfont[Ligatures=TeX]{Times New Roman}
\setsansfont{Arial}

% \usepackage[utf8x]{inputenc}
% \usepackage[mongolian]{babel}
%\usepackage{natbib}
\usepackage{geometry}
%\usepackage{fancyheadings} fancyheadings is obsolete: replaced by fancyhdr. JL
\usepackage{fancyhdr}
\usepackage{float}
\usepackage{afterpage}
\usepackage{graphicx}
\usepackage{amsmath,amssymb,amsbsy}
\usepackage{dcolumn,array}
\usepackage{tocloft}
\usepackage{dics}
\usepackage{nomencl}
\usepackage{upgreek}
\newcommand{\argmin}{\arg\!\min}
\usepackage{mathtools}
\usepackage[hidelinks]{hyperref}

\usepackage{algorithm}
\usepackage{algpseudocode}
\usepackage{color}
\definecolor{codegreen}{rgb}{0,0.6,0}
\definecolor{codegray}{rgb}{0.5,0.5,0.5}
\definecolor{codepurple}{rgb}{0.58,0,0.82}
\definecolor{backcolour}{rgb}{0.99,0.99,0.99}
\definecolor{lightgray}{rgb}{.9,.9,.9}
\definecolor{darkgray}{rgb}{.4,.4,.4}
\definecolor{purple}{rgb}{0.65, 0.12, 0.82}

\usepackage{listings}
\DeclarePairedDelimiter\abs{\lvert}{\rvert}%
\makeatletter
\usepackage{caption}
\captionsetup[table]{belowskip=0.5pt}
\usepackage{subfiles}


\usepackage{listings}
\renewcommand{\lstlistingname}{Код}
\renewcommand{\lstlistlistingname}{\lstlistingname ын жагсаалт}
\lstdefinelanguage{JavaScript}{
  keywords={typeof, new, true, false, catch, function, return, null, catch, switch, var, if, in, while, do, else, case, break, const, let},
  keywordstyle=\color{purple}\bfseries,
  ndkeywords={class, export, boolean, default, throw, implements, import, this},
  ndkeywordstyle=\color{blue}\bfseries,
  identifierstyle=\color{black},
  sensitive=false,
  comment=[l]{//},
  morecomment=[s]{/*}{*/},
  commentstyle=\color{lightgray}\ttfamily,
  morestring=[b]',
  morestring=[b]"
}

\lstset{
   language=JavaScript,
   backgroundcolor=\color{lightgray},
   extendedchars=true,
   basicstyle=\linespread{0.8}\small\ttfamily
   showstringspaces=false,
   showspaces=false,
   numbers=left,
   numberstyle=\footnotesize,
   numbersep=9pt,
   tabsize=3,
   breaklines=true,
   showtabs=false,
   captionpos=b,
   columns=fullflexible
}



\lstdefinestyle{mystyle}{
    basicstyle=\ttfamily\small,
    backgroundcolor=\color{backcolour},
    commentstyle=\color{codegreen},
    keywordstyle=\color{magenta},
    numberstyle=\tiny\color{codegray},
    stringstyle=\color{codepurple},
    %basicstyle=\footnotesize,
    breakatwhitespace=false,
    breaklines=true,
    captionpos=b,
    keepspaces=false,
    numbers=left,
    numbersep=10pt,
    showspaces=false,
    showstringspaces=true,
    showtabs=false,
    tabsize=2
}

\lstset{style=mystyle, label=DescriptiveLabel}

\let\oldabs\abs
\def\abs{\@ifstar{\oldabs}{\oldabs*}}
\makenomenclature
\begin{document}


%----------------------------------------------------------------------------------------
%   Өөрийн мэдээллээ оруулах хэсэг
%----------------------------------------------------------------------------------------

% Дипломийн ажлын сэдэв
\title{Веб аппликейшн хөгжүүлэлт}
% Дипломын ажлын англи нэр
\titleEng{Web application development}
% Өөрийн овог нэрийг бүтнээр нь бичнэ
\author{Энхбаярын Жавхлан}
% Өөрийн овгийн эхний үсэг нэрээ бичнэ
\authorShort{Э.Жавхлан}
% Удирдагчийн зэрэг цол овгийн эхний үсэг нэр
\supervisor{Т.Билгүүн, Фронт-энд хөгжүүлэгч}
% Хамтарсан удирдагчийн зэрэг цол овгийн эхний үсэг нэр
\cosupervisor{Д.Цолмон}
\company{"Зочил Технологи" ХХК}
% СиСи дугаар
\sisiId{20B1NUM0649}
% Их сургуулийн нэр
\university{МОНГОЛ УЛСЫН ИХ СУРГУУЛЬ}
% Бүрэлдэхүүн сургуулийн нэр
\faculty{МЭДЭЭЛЛИЙН ТЕХНОЛОГИ, ЭЛЕКТРОНИКИЙН СУРГУУЛЬ}
% Тэнхимийн нэр
\department{МЭДЭЭЛЭЛ, КОМПЬЮТЕРИЙН УХААНЫ ТЭНХИМ}
% Зэргийн нэр
\degreeName{үйлдвэрийн дадлагын ажлын тайлан}
% Суралцаж буй хөтөлбөрийн нэр
\programeName{Програм хангамж}
% Хэвлэгдсэн газар
\cityName{Улаанбаатар хот}
% Хэвлэгдсэн огноо
\gradyear{2024 он}


%----------------------------------------------------------------------------------------
%   Доорх хэсгийг өөрчлөх шаардлагагүй
%----------------------------------------------------------------------------------------
\include{src/main-pre}

% Удиртгалыг оруулж ирэх ба abstract.tex файлд удиртгалаа бичнэ
\begin{abstract}
Миний бие \@author \ үйлдвэрийн дадлагын ажлыг “Зочил технологи” ХХК  компани дээр гүйцэтгэсэн. Энэхүү үйлдвэрийн дадлагын хүрээнд хичээлээс болон бие даан эзэмшсэн мэдлэг, чадвараа ашиглан бодит ажлын орчинд бүтээгдэхүүн хөгжүүлж бодит хэрэглэгчдийн гарт хүргэх процесст суралцан Админ веб удирдлагын систем төслийн front-end хөгжүүлэлтэнд оролцсон.

\textbf{Зорилго:} React, Bootstrap технологиудын талаар судалж бүтээгдэхүүн хөгжүүлэх, компанийн хөгжүүлэлтийн арга барилтай танилцах.

\textbf{Зорилт:} Удирдагчийн зааварчилгааны дагуу алхам алхмаар судалгаа хийж өгсөн шаардлагын хүрээнд хэрэгжүүлэлт хийх. Хурал уулзалтуудад хамрагдаж, бодит бүтээгдэхүүн хөгжүүлэлтийн процесст оролцох. Back-end хөгжүүлэлтийн багтай багаар хамтран ажиллах.

\end{abstract}


%----------------------------------------------------------------------------------------
%   Дипломын үндсэн хэсэг эндээс эхэлнэ
%----------------------------------------------------------------------------------------
%\addcontentsline{toc}{part}{БҮЛГҮҮД}
% Шинэ бүлэг

\begin{table}[h]
	\caption{Дадлагын ажлын төлөвлөгөө}
	\begin{tabular}{|p{0.5cm}|p{8cm}|l|l|p{3cm}|}
	\hline
	\textbf{№} & \textbf{Гүйцэтгэх ажил} & \textbf{Хугацаа} & \textbf{Биелэлт} & \textbf{Дадлагын удирдагчийн үнэлгээ} \\ \hline
	1 & Zochil платформын Админ веб болон гар утасны Zochil IO апп-г судлах & 12/29 - 01/02 && \\ \hline
	2 & Админ веб дээр бараа бүтээгдэхүүн болон захиалгуудын жагсаалтын responsive засах & 01/02 - 01/04 && \\ \hline
	3 & React-hook-form-г судлах & 01/04 - 01/05 && \\ \hline
	4 &  Олон төрлийн динамик баннерийн CRUD форм хийх  & 01/05 - 01/07 && \\ \hline
	5 &  Арга хэмжээ үүсгэх CRUD форм хийх & 01/07 - 01/09 && \\ \hline
	6 & Zochil платформын үйлчилгээний төлбөрийн саналын хуудас хийх & 01/09 - 01/11 && \\ \hline
	8 & Мод бүтэцтэй бүтээгдэхүүний категори нэмэх компонент хийх  & 01/11 - 01/14 && \\ \hline
   9 & Zochil платформын гарын авлагыг GitBook дээр шинэчлэн хийх  & 01/14 - 01/20 && \\ \hline
   % 10 & Zochil платформын Landing page-г Webflow ашиглан шинэчлэн хийх & 01/17- 01/24 && \\ \hline
   11 & Админ вебийн зарим хуудсын responsive засах & 01/20- 01/29 && \\ \hline

	\end{tabular}
\end{table}
\chapter{Онолын судалгаа}
\subfile{src/chapters/chapter1.tex}
\chapter{Системийн шинжилгээ, зохиомж}
\subfile{src/chapters/chapter2.tex}
\chapter{Системийн хэрэгжүүлэлт}
\subfile{src/chapters/chapter3.tex}

%----------------------------------------------------------------------------------------
%   Дүгнэлт эндээс эхэлнэ
%----------------------------------------------------------------------------------------
\chapter{Дүгнэлт}
\indent Бакалаврын судалгааны ажлын хүрээнд блокчэйн технологи болон дижитал эрхийн менежментийн талаар судалсан. Энэхүү судалж суралцсан мэдлэгээ ашиглан практикт цахим бүтээлийн лицензийн системийг бүтээхээр зорьсон. Уг системийн гол зорилго нь хэрэглэгчид блокчэйн технологиор дамжуулан цахим бүтээлийг хамгаалах, хуваалцах, түүнд хандах зөвшөөрөл олгох төвлөрсөн бус, ил тод систем юм.
\\ \indent Энэхүү системийг эхнээс нь алхам алхмаар шаардлагаас нь хэрэгжүүлэлт хүртэл хийж гүйцэтгэсэн ба системийн функциональ ба функциональ бус шаардлагыг боловсруулж түүнээс системийн статик болон динамик загварыг гаргасан билээ. Үр дүнд нь блокчэйн, ухаалаг гэрээ болон IPFS технологиудыг ашиглан найдвартай, ил тод, төвлөрсөн бус системийг бүтээлээ.


%----------------------------------------------------------------------------------------
%   Дипломын номзүй, хавсралтын хэсэг эндээс эхэлнэ
%----------------------------------------------------------------------------------------

\singlespace
\addcontentsline{toc}{part}{НОМ ЗҮЙ}
\begin{thebibliography}{}
	% Ашигласан материалыг эндээс оруулна
   \bibitem{blockchain}
   Adam Hayes, Blockchain Facts: What Is It, How It Works, and How It Can Be Used. (December 15, 2023) \url{https://www.investopedia.com/terms/b/blockchain.asp}
   \bibitem{dlt}
   Scott Nevil, Distributed Ledger Technology (DLT): Definition and How It Works. (May 31, 2023) \url{https://www.investopedia.com/terms/d/distributed-ledger-technology-dlt.asp}
   \bibitem{zug_digtal_id}
    Sundararajan S. UN Agencies Turn to Blockchain In Fight Against Child Trafficking. (Nov 13, 2017)  \url{https://www.coindesk.com/markets/2017/11/13/un-agencies-turn-to-blockchain-in-fight-against-child-trafficking/}
   \bibitem{zug_digtal_id}
   Zug Digital ID: Blockchain Case Study for Government Issued Identity.  \url{https://www.investopedia.com/terms/b/blockchain.asp}
   \bibitem{drm}
   What is digital rights management (DRM)?.  \url{https://business.adobe.com/blog/basics/digital-rights-management}

\end{thebibliography}
\appendix
\addcontentsline{toc}{part}{ХАВСРАЛТ}

% Хавсралтын нэр. Хавсралт гэдэг үг агуулахгүй
\chapter{Үечилсэн төлөвлөгөө}
\begin{figure}[h!]
   \centering
   \includegraphics[scale=0.065, angle=90]{src/images/periodic-plan.png}
   \caption{Удирдагчийн үнэлгээ дүгнэлт}
\end{figure}


\chapter{Кодын хэрэгжүүлэлт}
\lstinputlisting[language=TypeScript, caption=Ухаалаг гэрээ,basicstyle=\linespread{0.8}\ttfamily,frame=single]{src/code/smart-contract.sol}

Уг код нь хэрэглэгчийн оруулах цахим бүтээлийг  мэдээллийг блокчэйнд бичнэ.
\lstinputlisting[language=TypeScript, caption=Блокчэйнд бичих,basicstyle=\linespread{0.8}\ttfamily,frame=single]{src/code/writeFile.ts}

Энэ функц нь хэрэглэгчийн оруулсан баримт бичгийг IPFS-д байршуулна.
\lstinputlisting[language=TypeScript, caption=Файл IPFS-д  байршуулах,basicstyle=\linespread{0.8}\ttfamily,frame=single]{src/code/uploadIpfs.ts}

Уг код нь хэрэглэгчийн оруулсан цахим баримт бичгүүдийн мэдээллийг блокчэйнээс уншина.
\lstinputlisting[language=TypeScript, caption=Блокчэйнээс унших,basicstyle=\linespread{0.8}\ttfamily,frame=single]{src/code/getUserFiles.ts}



%----------------------------------------------------------------------------------------
%   Хавсралтууд эндээс эхэлнэ
%----------------------------------------------------------------------------------------

\end{document}
